\documentclass[12pt,a4paper]{article}

\usepackage[T1]{fontenc}
\usepackage[utf8]{inputenc}
\usepackage{ctex}
\usepackage{amsmath,amsfonts,amssymb,amsthm}
\usepackage{abstract,appendix,natbib,caption}
\usepackage{makeidx,hyperref,url,booktabs}
\usepackage{graphicx,epsfig,subfig}
\usepackage{geometry}
\usepackage{xcolor}

\newtheorem{theorem}{定理}

\geometry{scale=0.8}

%\setlength{\lineskip}{\baselineskip}
\setlength{\parskip}{0.5\baselineskip}

\title{来自灵魂深处的问题}
\author{sis-flag}
\date{\today}

\begin{document}

%区域$\Omega = [0,1]^d$,维数$d = 1, 2$。
%
%一维的情况时,区域$\Omega=[0,1]$被均匀分成$M$个区间,区域$\Omega=[0,1] \times [0,1]$被均匀分成$M \times M$个方格。在每个方格上$V(x)$是分片常数,而且大于等于0。
%
%特征值问题
%\begin{align}
%- \triangle u(x) + V(x) u(x) = \lambda u(x) \qquad x \in \Omega
%\end{align}
%导数边界条件。参数$h \geq 0$,$\mathbf{n}$代表外法向量。
%\begin{align}
%\frac{\partial u}{\partial \mathbf{n}}(x) + h u(x) = 0 \qquad x \in \partial \Omega
%\end{align}
%
%定义$w(x)$是这个方程的解,边界条件相同。
%\begin{align}
%- \Delta w(x) + V(x) w(x) = 1 \qquad x \in \Omega
%\end{align}
%
%\vline
%
%\begin{theorem}
%
%在特征值问题中,特征值$\lambda_k$,对应特征函数$u_k(x)$,满足$\max_{x \in \Omega} |u_k(x)| = 1$。则
%
%固定边界条件下,它们满足
%\begin{align}
%|u_k(x)| \leq \lambda_k w(x) \qquad x \in \Omega
%\end{align}
%
%\end{theorem}
%
%\paragraph{证明}
%
%方程的格林函数
%\begin{align}
%(-\triangle + V(x)) G_y(x) = \delta_y(x) \qquad x,y \in \Omega \\
%\end{align}
%其中$\delta_y(x)$是Delta函数。
%
%根据格林函数的性质,非负$G(x,y) \geq 0$。
%
%定义双线性泛函
%$$ B[u,v] = \int_\Omega \nabla u \cdot \nabla v \ dx + h \int_{\partial \Omega} V \, u \, v \ dx $$
%显然它对称$B[u,v] = B[v,u]$
%
%以上问题的弱形式可以写为
%\begin{align}
%B[u_k, v] = \lambda_k \int_\Omega u_k \, v \ dx \qquad \forall v \in H_1(\Omega) \label{e1} \\
%B[w, v] = \int_\Omega 1 \cdot v \ dx \qquad \forall v \in H_1(\Omega) \label{e2} \\
%B[G_y, v] = v(y) \ dx \qquad \forall v \in H_1(\Omega) \label{e3}
%\end{align}
%
%(\ref{e1})中测试函数$v$取为$G_y$,(\ref{e3})中$v$取为$u_k$得到
%$$ \lambda_k \int_\Omega G_y(x) u_k(x) = B[u_k, G_y] = B[G_y, u_k] = u_k(y) $$
%
%类似,(\ref{e2})中测试函数$v$取为$G_y(x)$,(\ref{e3})中$v$取为$w$得到
%$$ \int_\Omega G_y(x) \cdot 1 = B[w, G_y] = B[G_y, w] = w(y) $$
%
%就得到了
%\begin{align}
%w(y) = \int_\Omega G_y(x) \cdot 1 \ dx \qquad u_k(y) = \lambda_k \int_\Omega G_y(x) u_k(x) \ dx
%\end{align}
%
%此时有
%$$ u_k(y) \leq \lambda_k \int_\Omega |G_y(x) u_k(x)| \ dx \leq \lambda_k \max_{x \in \Omega} |u_k(x)| \int_\Omega G_y(x) \ dx = \lambda_k \, \max_{x \in \Omega} |u_k(x)| w(y)$$
%
%归一化$\max_{x \in \Omega} |u_k(x)| = 1$的情况下就是
%\begin{align}
%|u_k(x)| \leq \lambda_k w(x) \qquad x \in \Omega
%\end{align}



\maketitle

区域$\Omega = [0,1]^d$,维数$d = 1, 2, 3$。

区域上的二阶非对称椭圆算子
\begin{align*}
L u = -\nabla(A \nabla u) + b \cdot \nabla u + c u
\end{align*}
满足非退化,以及解的存在唯一性,适定性等条件。还要满足极值原理的条件。

它的共轭算子为
\begin{align*}
L^{*} u & = -\nabla(A \nabla u) - \nabla (b u) + c u \\
& = -\nabla(A \nabla u) - b \cdot \nabla u + (c - \nabla b) u
\end{align*}

特征值问题
\begin{align*}
L u = \lambda u \qquad x \in \Omega \\
n^T A \nabla u + h u = 0 \qquad x \in \partial \Omega
\end{align*}
其中$h(x) \geq 0$,$n$是边界的外法向量。

$w(x)$是这个方程的解,边界条件相同。
\begin{align*}
L w = 1 \qquad x \in \Omega \\
n^T A \nabla w + h w = 0 \qquad x \in \partial \Omega
\end{align*}

定义格林函数
\begin{align*}
L^{*} G_y = \delta_y \qquad x \in \Omega \\
n^T A \nabla G_y + (h + b \cdot n) G_y = 0 \qquad x \in \partial \Omega
\end{align*}

假设函数$u$满足原问题的边界条件,$v$满足格林函数对应的边界条件,根据分部积分公式(写成分量的形式)
\begin{align*}
(L u, v) = \int_\Omega [-\sum_{i,j} \partial_i (a_{i,j} \partial_j u) + \sum_{i} b_i \partial_i u + c u] v \ dx
\end{align*}
其中
\begin{align*}
& \int_\Omega - \partial_i (a_{i,j} \partial_j u) v \ dx \\
= & \int_\Omega a_{i,j} \, \partial_j u \, \partial_i v \ dx - \int_{\partial \Omega} n_i a_{i,j} (\partial_j u) v \ ds \\
= & \int_\Omega - \partial_j (a_{i,j} \partial_i v) u \ dx + \int_{\partial \Omega} n_j a_{i,j} (\partial_i v) u \ ds - \int_{\partial \Omega} n_i a_{i,j} (\partial_j u) v \ ds
\end{align*}
\begin{align*}
\int_\Omega b_i \, \partial_i u \, v \ dx = \int_\Omega - \partial_i (b_i \, v) \, u \ dx + \int_{\partial \Omega} n_i \, b_i \, u \, v \ ds
\end{align*}

得到
\begin{align*}
\int_\Omega [-\sum_{i,j} \partial_i (a_{i,j} \partial_j u) + \sum_{i} b_i \partial_i u + c u] v \ dx = & \int_\Omega [-\sum_{i,j} \partial_j (a_{i,j} \partial_i v) - \sum_{i} \partial_i (b_i v) + c v] u \ dx \\
+ & \int_{\partial \Omega} \sum_{i} n_i \, b_i \, u \, v \ ds \\
- & \int_{\partial \Omega} \sum_{i,j} n_i \, a_{i,j} \, (\partial_j u) \, v \ ds \\
+ & \int_{\partial \Omega} \sum_{i,j} n_j \, a_{i,j} \, (\partial_i v) \, u \ ds
\end{align*}

边界条件按分量的形式写出来是
\begin{align*}
\sum_{i,j} n_i \, a_{i,j} \, (\partial_j u) + h \, u = 0
\end{align*}
\begin{align*}
\sum_{i,j} n_j \, a_{i,j} \, (\partial_i v) + \sum_{i} n_i \, b_i \, v + h \, v = 0
\end{align*}
把这两个条件代入,边界上的积分就消失了。

于是得到
\begin{align*}
 (u, L^{*} G_y) = (L u, G_y) \qquad (w, L^{*} G_y) = (L w, G_y)
\end{align*}

所以有
\begin{align*}
u(y) = (u, \delta_y) = (u, L^{*} G_y) = (L u, G_y) = \lambda (u, G_y) \\
w(y) = (w, \delta_y) = (w, L^{*} G_y) = (L w, G_y) = (1, G_y)
\end{align*}

如果格林函数有非负性
\begin{align*}
|u(y)| = |\lambda| \ |\int_\Omega G_y u \ dx| \leq |\lambda| \int_\Omega |G_y u| \ dx = |\lambda| \int_\Omega |u| G_y \ dx
\end{align*}
归一化$\max_{x \in \Omega} |u(x)| = 1$的情况下
\begin{align*}
|u(y)| \leq |\lambda| \int_\Omega 1 \cdot G_y \ dx = |\lambda| w(y)
\end{align*}

目前还没有在数值上找到反例。。。

\end{document}