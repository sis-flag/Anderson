\documentclass[UTF8,12pt]{article}

\usepackage[utf8]{inputenc}
\usepackage{ctex}
\usepackage{amsmath,amsfonts,amssymb}
\usepackage{graphicx,epsfig,subfig}
\usepackage{makeidx,hyperref}
\usepackage{geometry}
\usepackage{listings}
\usepackage[linesnumbered,boxed]{algorithm2e}
\usepackage{xcolor}

\geometry{scale=0.8}

%\setlength{\lineskip}{\baselineskip}
\setlength{\parskip}{0.5\baselineskip}

\title{Anderson局部化实验报告4}
\author{flag}
\date{\today}

\begin{document}
    
\maketitle

定义算子$L = -\Delta + K V$,它是一个正的,对称的算子。

研究特征值问题
\begin{eqnarray}\label{e0}
L u = \lambda u \qquad \mathbf{x} \in \Omega \\
\frac{\partial u}{\partial n} + h u = 0 \qquad \mathbf{x} \in \partial \Omega
\end{eqnarray}

对某个特定的特征值$\lambda$和对应的特征函数$u$,在子区域$\Omega_1$上,假设我们知道$u$在边界$\partial \Omega_1$上的值是$g = u|_{\partial \Omega_1}$,则$u$在内部的值可以看作这样一个方程的解。
\begin{eqnarray}
L u - \lambda u = 0 \qquad \mathbf{x} \in \Omega_1 \\
\frac{\partial u}{\partial n} + h u = 0 \qquad \mathbf{x} \in \partial \Omega_1 \cap \partial \Omega \\
u = g \qquad \mathbf{x} \in \partial \Omega_1 \setminus \partial \Omega
\end{eqnarray}

在子区域$\Omega_1$上,定义函数$v$是一个边界条件相同的方程的解。
\begin{eqnarray}
L v = 0 \qquad \mathbf{x} \in \Omega_1 \\
\frac{\partial v}{\partial n} + h v = 0 \qquad \mathbf{x} \in \partial \Omega_1 \cap \partial \Omega \\
v = g \qquad \mathbf{x} \in \partial \Omega_1 \setminus \partial \Omega
\end{eqnarray}

此时$w = u - v$在边界上就是零边界条件。

定义空间$H_{h,0}(\Omega_1) = \{ w \in H^1(\Omega_1) | \frac{\partial w}{\partial n} + h w = 0 \; \text{for} \; \partial \Omega_1 \cap \partial \Omega \; \text{and} \; u = 0 \; \text{for} \; \partial \Omega_1 \setminus \partial \Omega \}$。它是$H^1(\Omega_1)$的一个线性子空间。$w$就在这个空间$H_{h,0}(\Omega_1)$里面。

考虑特征值问题$L \phi = \lambda \phi$,它的特征函数$\phi_1, \phi_2, \cdots$构成空间的一组正交的Hilbert基。把$w$按这组基函数展开,得到
%$$ w = \sum_k w_k \phi_k \qquad w_k = \int_{\Omega_1} w \phi_k \ dx $$
$$ w = \sum_k w_k \phi_k $$
此时有
$$ L w = \sum_k \lambda_k w_k \phi_k \quad \text{and} \quad L w - \lambda w = \sum_k (\lambda_k - \lambda) w_k \phi_k $$
就得到
$$ \| L w - \lambda w \|_{L^2(\Omega_1)}^2 = \sum_k (\lambda_k - \lambda)^2 w_k^2 \geq \min_k (\lambda_k - \lambda)^2\sum_k w_k^2 = dist(\lambda, \lambda
(\Omega_1))^2 \|w\|_{L^2(\Omega_1)}^2 $$
注意到$L u - \lambda u = 0, L v = 0$,就是$L w - \lambda w = L(u - v) - \lambda(u - v) = \lambda v$。

记$d = dist(\lambda, \lambda(\Omega_1))$因此得到
\begin{equation}
\lambda \| v \|_{L^2(\Omega_1)} \geq d \|w\|_{L^2(\Omega_1)}
\end{equation}
就是
\begin{equation}
\|u\|_{L^2(\Omega_1)} \leq \|v\|_{L^2(\Omega_1)} + \|w\|_{L^2(\Omega_1)} \leq (1 + \frac{\lambda}{d}) \|v\|_{L^2(\Omega_1)}
\end{equation}

文章附录中的证明到这里就结束了。

\vline

上面的证明就是把文章附录里的证明推广到Robin边界条件的情况。但是在把定理应用到文章中的结论的时候,不同的边界就不一样了。

在Dirichlet边界条件下,得到$v$满足的方程
\begin{eqnarray}
L v = 0 \qquad \mathbf{x} \in \Omega_1 \\
v = u|_{\partial \Omega_1} \qquad \mathbf{x} \in \partial \Omega_1
\end{eqnarray}
这个时候,$\|v\|_{L^2(\Omega_1)}$可以被边界$\|u|_{\partial \Omega_1}\|$控制,之后又因为landscape的控制关系$\|u|_{\partial \Omega_1}\|$可以被landscape在边界$\partial \Omega_1$的范数控制。所以在landscape在边界处很小的时候,可以用特征值是否相近来判断特征函数是否localize到这一块区域。

在Robin边界条件下,$v$满足的方程是
\begin{eqnarray}\label{q1}
L v = 0 \qquad \mathbf{x} \in \Omega_1 \\
\frac{\partial v}{\partial n} + h v = 0 \qquad \mathbf{x} \in \partial \Omega_1 \cap \partial \Omega \\
v = g \qquad \mathbf{x} \in \partial \Omega_1 \setminus \partial \Omega
\end{eqnarray}
这个时候,$\|v\|_{L^2(\Omega_1)}$不仅仅被边界$\|u|_{\partial \Omega_1}\|$影响,还受到边界条件中的参数$h$的影响。即使landscape在子区域在内部的边界上很小,也不能说明$\|v\|_{L^2(\Omega_1)}$在子区域的内部很小。条件就失效了。

所以说现在的问题就是要用$g$控制$v$,我们用极值原理来处理它。

假设方程\ref{q1}有经典解。\textbf{而且h>0}。由于$L = -\Delta + KV$中的$KV \geq 0$,方程的解满足极值原理,最大值一定在边界$\partial \Omega_1$处取到。

设最大值在$x_0$处取到,如果$x_0 \in \partial \Omega_1 \cap \partial \Omega $,说明最大值在Dirichlet边界上取到,就有$\|v\|_{L^\infty(\Omega_1)} < \max_{x \in \partial \Omega_1 \cap \partial \Omega} g(x)$,就可以控制了。

如果$x_0 \in \partial \Omega_1 \setminus \partial \Omega $,说明最大值在Robin边界上取到,由于$g > 0$,最大值应该是大于0的数。根据Robin边界条件$\frac{\partial v}{\partial n} + h v = 0$,而且$h > 0$,可以得到$\frac{\partial v}{\partial n} < 0$。这说明解沿外法方向是递减的,这和解的最大值在边界取到矛盾。

综上,我们就得到了在$\Omega_1$上,$$\|v\|_{L^\infty(\Omega_1)} < \max_{x \in \partial \Omega_1 \cap \partial \Omega} g(x)$$

\vline

\textbf{最终结论}

设$u$是特征值问题\ref{e0}对应于特征值$\lambda$的解,满足$\|u\|_{L^\infty(\Omega)} = 1$,参数$h > 0$。在子区域$\Omega_1$上,就有
$$ \|u\|_{L^2(\Omega_1)} \leq C (1 + \frac{\lambda}{dist(\lambda, \lambda(\Omega_1))}) \max_{x \in \partial \Omega_1 \cap \partial \Omega} u(x) $$
其中$C$是只和$\Omega_1$的测度有关的常数。

进一步地,如果$w$是对应的landscape,满足$u(x) \leq (\lambda + \alpha + \beta) w(x)$,则有
$$ \|u\|_{L^2(\Omega_1)} \leq C (\lambda + \alpha + \beta) (1 + \frac{\lambda}{dist(\lambda, \lambda(\Omega_1))}) \max_{x \in \partial \Omega_1 \cap \partial \Omega} w(x) $$

\end{document}